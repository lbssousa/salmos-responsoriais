–1 Ó Senhor, por que ficais assim tão longe, *
e, no tempo da aflição, vos escondeis,
–2 enquanto o pecador se ensoberbece, *
o pobre sofre e cai no laço do malvado?

–3 O ímpio se gloria em seus excessos, *
blasfema o avarento e vos despreza;
–4 em seu orgulho ele diz: “Não há castigo! *
Deus não existe!”  5É isto mesmo que ele pensa.

= Prospera a sua vida em todo tempo; †
vossos juízos estão longe de sua mente; *
ele vive desprezando os seus rivais.
–6 No seu íntimo ele pensa: “Estou seguro! *
Nunca jamais me atingirá desgraça alguma!”

–7 Só há maldade e violência em sua boca, *
em sua língua, só mentira e falsidade.
–8 Arma emboscadas nas saídas das aldeias, *
mata inocentes em lugares escondidos.

–9 Com seus olhos ele espreita o indefeso, *
como um leão que se esconde atrás da moita;
– assalta o homem infeliz para prendê-lo, *
agarra o pobre e o arrasta em sua rede.

–10 Ele se curva, põe-se rente sobre o chão, *
e o indefeso tomba e cai em suas garras.
–11 Pensa consigo: “O Senhor se esquece dele, *
esconde o rosto e já não vê o que se passa!”

–12 Levantai-vos, ó Senhor, erguei a mão! *
Não esqueçais os vossos pobres para sempre!
–13 Por que o ímpio vos despreza desse modo? *
Por que diz no coração: “Deus não castiga?”

–14 Vós, porém, vedes a dor e o sofrimento, *
vós olhais e tomais tudo em vossas mãos!
– A vós o pobre se abandona confiante, *
sois dos órfãos vigilante protetor.

–15 Quebrai o braço do injusto e do malvado! *
Castigai sua malícia e desfazei-a!
–16 Deus é Rei durante os séculos eternos. *
Desapareçam desta terra os malfeitores!

–17 Escutastes os desejos dos pequenos, *
seu coração fortalecestes e os ouvistes,
=18 para que os órfãos e oprimidos deste mundo †
tenham em vós o defensor de seus direitos, *
e o homem terreno nunca mais cause terror!