% chktex-file 1
% chktex-file 19
\newcommand{\GR}{\emph{Graduale Romanum}}
\newcommand{\GS}{\emph{Graduale Simplex}}
\newcommand{\KS}{\emph{Kyriale Simplex}}
\newcommand{\Al}{\emph{Allelúia}}
\newcommand{\Schola}{\emph{schola}}
\newcommand{\ScholaC}{\emph{schola cantorum}}
\newcommand{\FirstPara}[1]{\noindent\textcolor{gregoriocolor}{#1.}}
\newcommand{\Para}[1]{\vspace{\baselineskip}\noindent\textcolor{gregoriocolor}{#1.}}

\tocchapter{Introdução}

\tocsection{Sobre este projeto}

\subsection{Motivação}

Este caderno é uma extensão do projeto de adaptação do {\GS} para língua portuguesa a partir dos textos litúrgicos oficiais do Brasil. A proposta deste caderno é oferecer uma forma de aplicar os tons salmódicos arcaicos utilizados no {\GS} para cantos interlecionais (salmos responsoriais e aclamações ao Evangelho) aos textos do Lecionário. Esta é uma importante lacuna a ser preenchida, visto que nem sempre os salmos responsoriais e aleluiáticos do próprio {\GS} são viáveis para utilização nas comunidades paroquiais.

\subsection{Desafios para a elaboração deste caderno}

No que tange aos cantos de aclamação ao Evangelho, o processo de acomodação melódica aos textos do Lecionário é trivial, bastando definir um critério para a escolha dos tons salmódicos e aplicá-los diretamente ao texto. Para os salmos responsoriais, por sua vez, o processo de adaptação dos tons salmódicos arcaicos aos textos do Lecionário foi bem mais desafiador, principalmente devido à grande diferença na forma dos salmos responsoriais entre as duas fontes:
\begin{itemize}
  \item nos salmos responsoriais do {\GS}, canta-se um responsório breve ao final de cada verso salmódico (que geralmente corresponde a um versículo do salmo em questão).
  \item no Lecionário, inicia-se o canto pelo refrão, que pode ser mais breve ou mais longo, semelhante a uma antífona, ao qual se alternam os versos salmódicos agrupados em estrofes.
\end{itemize}

O maior desafio para este projeto foi tomar as melodias dos refrões breves do {\GS} e estendê-las, de maneira coerente, para que pudessem ser acomodadas aos refrões de tamanho variável do Lecionário. Para isso, analisamos todas as variações de refrões presentes no {\GS} para cada tom salmódico, extraímos os padrões melódicos comuns todas elas e, sobre estes padrões, elaboramos uma fórmula geral que pudesse ser adaptada a diferentes tamanhos de texto.

Outro desafio, um pouco menor, foi estudar todos os tons salmódicos para salmos responsoriais do {\GS} e encontrar uma maneira de acomodá-los ao formato de versos salmódicos agrupados em estrofes, sem que o resultado final soasse estranho demais.

\subsection{Os tons salmódicos utilizados neste caderno}

Ao longo deste caderno, utilizaremos 13 tons salmódicos para salmos responsoriais, além de 3 tons salmódicos específicos para salmos aleluiáticos, isto é, salmos responsoriais cujo refrão seja \emph{Aleluia, aleluia, aleluia}. A relação dos tons salmódicos aqui utilizados é a seguinte:
\begin{itemize}
  \item C \protect\GreStar, C 1, C 2 a, C 2 g, C 3 a, C 3 g, D 1 e, D 1 g, D \protect\GreStar, E 2 d, E 3, E 4 e E 5, idênticos aos tons salmódicos de mesmo nome no {\GS}.
  \item Tom salmódico aleluiático I, tal como aparece na página 157 do {\GS}.
  \item Tom salmódico aleluiático II, tal como aparece na página 297 do {\GS}.
  \item Tom salmódico aleluiático III, tal como aparece na página 454 do {\GS}.
\end{itemize}

\subsection{Critérios para a escolha dos tons salmódicos}

Para as aclamações ao Evangelho, o critério de escolha dos tons salmódicos é relativamente simples. Para cada missa, procuramos tomar, dentre as diversas opções de {\Al} disponíveis no {\GS}, a que está escrita no mesmo modo gregoriano do canto correspondente no {\GR} ({\Al} ou, no Tempo da Quaresma, \emph{Tractus}) para a missa em questão (exemplo: VIII modo para o I Domingo do Advento), exceto quando este for o V modo, que não possui equivalente no {\GS} (neste caso, tomaremos a melodia em VI modo). Quando houver duas ou mais opções de {\Al} no {\GS} para o mesmo modo gregoriano, alternamos entre uma e outra opção, dando preferência às melodias mais ornadas para os domingos e solenidades.

Para os salmos responsoriais, a escolha do tom salmódico segue os seguintes critérios:
\begin{enumerate}
  \item Caso o salmo responsorial em questão apareça originalmente no {\GS} como salmo responsorial ou aleluiático, escolhemos preferencialmente o mesmo tom salmódico na ocorrência em questão, com as seguintes substituições para os tons salmódicos não utilizados:
        \begin{center}
          \begin{tabular}{|c|c|}
            \hline
            tom salmódico original & tom salmódico substituto \\
            \hline
            C 4                    & C 1                      \\
            D 1 b                  & D 1 g                    \\
            E 1                    & E 5                      \\
            E 2 e                  & E 2 d                    \\
            E \GreStar             & E 5                      \\
            \hline
          \end{tabular}
        \end{center}
  \item Caso o salmo responsorial em questão apareça originalmente no {\GS} apenas acompanhando alguma antífona em um dos oito modos gregorianos, escolhemos preferencialmente um tom salmódico que se aproxime do modo em questão. Para esta tarefa, elaboramos a seguinte tabela de correspondência:
        \begin{center}
          \begin{tabular}{|c|c|}
            \hline
            modo & tom salmódico prefencial \\
            \hline
            I    & C 3 a, C 3 g ou E 4      \\
            II   & C \protect\GreStar       \\
            III  & E 2 d ou E 5             \\
            IV   & D 1 e                    \\
            V    & E 4                      \\
            VI   & C 2 a, C 2 g ou E 3      \\
            VII  & D 1 g                    \\
            VIII & D \protect\GreStar       \\
            \hline
          \end{tabular}
        \end{center}
  \item Para todos os outros casos, escolhemos livremente o tom salmódico, priorizando a melhor acomodação melódica para o refrão do salmo responsorial.
\end{enumerate}
Em todo caso, sempre que se repetir um mesmo salmo com o mesmo refrão, utilizaremos sempre o mesmo tom salmódico.

Ressaltamos, porém, que a escolha dos tons salmódicos apresentadas neste caderno não pretende ser taxativa. Pode-se adaptar livremente qualquer tom salmódico aqui oferecido a qualquer salmo responsorial.